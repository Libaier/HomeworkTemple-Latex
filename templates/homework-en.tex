\documentclass{article}

\usepackage{fancyhdr}
\usepackage{extramarks}
\usepackage{amsmath}
\usepackage{amsthm}
\usepackage{amsfonts}
\usepackage{tikz}
\usepackage[plain]{algorithm}
\usepackage{algpseudocode}
\usepackage{array,booktabs}
\usepackage{float}


\usetikzlibrary{automata,positioning}

%
% Basic Document Settings
%

\topmargin=-0.45in
\evensidemargin=0in
\oddsidemargin=0in
\textwidth=6.5in
\textheight=9.0in
\headsep=0.25in

\linespread{1.1}

\pagestyle{fancy}
\lhead{\hmwkAuthorName}
\chead{\hmwkClass\ (\hmwkClassInstructor\ \hmwkClassTime): \hmwkTitle}
\rhead{\firstxmark}
\lfoot{\lastxmark}
\cfoot{\thepage}

\renewcommand\headrulewidth{0.4pt}
\renewcommand\footrulewidth{0.4pt}

\setlength\parindent{0pt}

%
% Create Problem Sections
%

\newcommand{\enterProblemHeader}[1]{
    \nobreak\extramarks{}{Problem \arabic{#1} continued on next page\ldots}\nobreak{}
    \nobreak\extramarks{Problem \arabic{#1} (continued)}{Problem \arabic{#1} continued on next page\ldots}\nobreak{}
}

\newcommand{\exitProblemHeader}[1]{
    \nobreak\extramarks{Problem \arabic{#1} (continued)}{Problem \arabic{#1} continued on next page\ldots}\nobreak{}
    \stepcounter{#1}
    \nobreak\extramarks{Problem \arabic{#1}}{}\nobreak{}
}

\setcounter{secnumdepth}{0}
\newcounter{partCounter}
\newcounter{homeworkProblemCounter}
\setcounter{homeworkProblemCounter}{1}
\nobreak\extramarks{Problem \arabic{homeworkProblemCounter}}{}\nobreak{}

%
% Homework Problem Environment
%
% This environment takes an optional argument. When given, it will adjust the
% problem counter. This is useful for when the problems given for your
% assignment aren't sequential. See the last 3 problems of this template for an
% example.
%
\newenvironment{homeworkProblem}[1][-1]{
    \ifnum#1>0
        \setcounter{homeworkProblemCounter}{#1}
    \fi
    \section{Problem \arabic{homeworkProblemCounter}}
    \setcounter{partCounter}{1}
    \enterProblemHeader{homeworkProblemCounter}
}{
    \exitProblemHeader{homeworkProblemCounter}
}

%
% Homework Details
%   - Title
%   - Due date
%   - Class
%   - Section/Time
%   - Instructor
%   - Author
%

\newcommand{\hmwkTitle}{Homework\ \#3}
\newcommand{\hmwkDueDate}{November 30, 2014}
\newcommand{\hmwkClass}{Data Mining}
\newcommand{\hmwkClassTime}{712009H}
\newcommand{\hmwkClassInstructor}{Professor Y L}
\newcommand{\hmwkAuthorName}{Libaier}
\newcommand{\hmwkAuthorCode}{1234567}

%
% Title Page
%

\title{
    \vspace{2in}
    \textmd{\textbf{\hmwkClass:\ \hmwkTitle}}\\
    \normalsize\vspace{0.1in}\small{Due\ on\ \hmwkDueDate\ at 5:00pm}\\
    \vspace{0.1in}\large{\textit{\hmwkClassInstructor\ \hmwkClassTime}}
    \vspace{3in}
}

\author{\textbf{\hmwkAuthorName}}
\date{\textbf{\hmwkAuthorCode}}

\renewcommand{\part}[1]{\textbf{\large Part \Alph{partCounter}}\stepcounter{partCounter}\\}

%
% Various Helper Commands
%

% Useful for algorithms
\newcommand{\alg}[1]{\textsc{\bfseries \footnotesize #1}}

% For derivatives
\newcommand{\deriv}[1]{\frac{\mathrm{d}}{\mathrm{d}x} (#1)}

% For partial derivatives
\newcommand{\pderiv}[2]{\frac{\partial}{\partial #1} (#2)}

% Integral dx
\newcommand{\dx}{\mathrm{d}x}

% Alias for the Solution section header
\newcommand{\solution}{\textbf{\large Solution}}

% Probability commands: Expectation, Variance, Covariance, Bias
\newcommand{\E}{\mathrm{E}}
\newcommand{\Var}{\mathrm{Var}}
\newcommand{\Cov}{\mathrm{Cov}}
\newcommand{\Bias}{\mathrm{Bias}}

\begin{document}

\maketitle

\pagebreak

\begin{homeworkProblem}
Cluster the following 8 points into three clusters:
\\A1(2,10), A2(2,5), A3(8,4), A4(5,8), A5(7,5), A6(6,4), A7(1,2), A8(4,9)\\
The distance function is Euclidean distance. Suppose initially we assign A1, A4, A7 as the center of each cluster, respectively. 
Use the K-Means algorithm to show only
    \\
(1)The three cluster center after the first round execution
    \\\\
    \solution
    \\Step 1.
    \begin{align*} 
    K & =3\\
    Z_1(1)&=A1=(2\ 10)^t\\
    Z_2(1)&=A4=(5\ 8)^t\\
    Z_3(1)&=A7=(1\ 2)^t
    \end{align*}
    \\Step 2.
    \begin{align*} 
    \\||Ai-Z_1(1)||&<min(||Ai-Z_2(2)||,||Ai-Z_2(2)||),i=1
    \\||Ai-Z_2(1)||&<min(||Ai-Z_1(2)||,||Ai-Z_3(2)||),i=3,4,5,6,8
    \\||Ai-Z_3(1)||&<min(||Ai-Z_1(2)||,||Ai-Z_2(2)||),i=2,7
    \end{align*}
    \begin{align*} 
    \\S_1(1)&=\{A1\}
    \\S_2(1)&=\{A3,A4,A5,A6,A8\}
    \\S_3(1)&=\{A2,A7\}
    \end{align*}
    \\Step 3.Calculate the new cluster center
    \begin{align*} 
    \\Z_1(2)&=\frac{1}{N}\sum_{x \in S_1(1)} x=(2\ 10)^t
    \\Z_2(2)&=\frac{1}{N}\sum_{x \in S_2(1)} x=(6\  6)^t
    \\Z_3(2)&=\frac{1}{N}\sum_{x \in S_3(1)} x=(1.5\ 3.5)^t
    \end{align*}
    \\\\(2)The final three clusters
    \\
    \solution   
    \begin{align*} 
    \\S_1(1)&=\{A1,A8,A4\}
    \\S_2(1)&=\{A3,A5,A6\}
    \\S_3(1)&=\{A2,A7\}
    \end{align*}
\end{homeworkProblem}

\pagebreak

\begin{homeworkProblem}
    Perform AGNES clustering on the data set in Question 1. Show your results by drawing a dendrogram (each step merges two clusters with the minimum distance). The dendrogram should clearly show the order in which the points are merged. 
    \\\\
    \solution
  \begin{figure}[H]
  \centering
  \includegraphics[width=4in]{f2.eps}
  \caption{The dendrogram.}
\end{figure}
\end{homeworkProblem}
\pagebreak

\begin{homeworkProblem}
    1.	Production Recommendation
The data contains the following fields. Row 1 contains the headers.
% Table generated by Excel2LaTeX from sheet 'Sheet1'
\begin{table}[htbp]
  \centering
  \caption{Production Recommendation}
    \begin{tabular}{ll}
    \toprule
    \textbf{id} & a unique identification number \\
    \midrule
    \textbf{age} & age of customer in years \\
    \textbf{sex} & MALE / FEMALE \\
    \textbf{region} & inner\_city/rural/suburban/town \\
    \textbf{income} & income of customer \\
    \textbf{married} & is the customer married (YES/NO) \\
    \textbf{children} & number of children \\
    \textbf{car} & does the customer own a car (YES/NO) \\
    \textbf{save\_acct} & does the customer have a saving account (YES/NO) \\
    \textbf{current\_acct} & does the customer have a current account (YES/NO) \\
    \textbf{mortgage} & does the customer have a mortgage (YES/NO) \\
    \textbf{pep} & did the customer buy a PEP after the last mailing (YES/NO) \\
    \bottomrule
    \end{tabular}%
  \label{tab:addlabel}%
\end{table}%
\\Each record is a customer description where the "pep" field indicates whether or not that customer bought a PEP.  For other existing customers in the database, we would like to see if PEP should be RECOMMENDED to the customers in the roll-out data. 
    \\The firm decides to use decision tree to build the models for PEP recommendation.
Develop a decision tree model using the estimation data. For building this model, you are expected to use the following steps. 

Using the “bank-estimation-data”, estimate the decision tree that predicts pep as a function of the other variables. Select “Expert” and set “pruning severity” at 75. Set the “Type” of pep as “Flag” and the “Direction” as “out”. Build decision trees using three options “Minimum records per child branch” values being (a) 56, (b) 15 and (c) 10, not selecting “use global pruning”. 
\\\\1)	Hand in: the confusion matrix for (a), (b) and (c) on the “validation data”.
  \\\\
    \solution
  \begin{figure}[H]
  \centering
  \includegraphics[width=3in]{hw3a.eps}
  \caption{confusion matrix for (a).}
\end{figure}
 \begin{figure}[H]
  \centering
  \includegraphics[width=3in]{hw3b.eps}
  \caption{confusion matrix for (b).}
\end{figure}
 \begin{figure}[H]
  \centering
  \includegraphics[width=3in]{hw3c.eps}
  \caption{confusion matrix for (c).}
\end{figure}
2)	Hand in: Which of the three trees will you use to score the data in a holdout data list and why? 2-3 lines
\\\\
    \solution
    \\
    I will use the tree with minimum records per child branch of 15 for it's high recognition rate of the 1-1 state(The correct prediction for the one who will buy PEP).
    \\\\3)	Hand in: For the following data (Appendix 1), using the rules from the best decision tree, fill in the recommendation.
 \\\\
    \solution
% Table generated by Excel2LaTeX from sheet 'Sheet1'
\begin{table}[htbp]
  \centering
  \caption{Appendix 1}
    \begin{tabular}{cccccccccc}
    \toprule
    region & income & married & children & car   & save\_act & current\_act & mortgage & RECOMEND PEP \\
    \midrule
    1     & 14000 & 0     & 3     & 0     & 1     & 1     & 0     &F     &  \\
    0     & 33000 & 0     & 0     & 1     & 1     & 0     & 0     &T     &  \\
    0     & 16700 & 1     & 1     & 0     & 1     & 1     & 0     &F     &  \\
    1     & 43400 & 1     & 1     & 1     & 1     & 1     & 0     &T     &  \\
    2     & 60000 & 1     & 1     & 0     & 1     & 1     & 0     &T     &  \\
    0     & 27700 & 0     & 1     & 1     & 0     & 0     & 0     &T     &  \\
    0     & 38784 & 1     & 0     & 0     & 1     & 1     & 0     &F     &  \\
    0     & 10200 & 1     & 0     & 0     & 1     & 1     & 1     &F     &  \\
    0     & 22000 & 1     & 1     & 1     & 1     & 0     & 1     &T     &  \\
    1     & 37400 & 1     & 2     & 0     & 1     & 1     & 0     &T     &  \\
    \bottomrule
    \end{tabular}%
  \label{tab:addlabel}%
\end{table}%
\end{homeworkProblem}
\pagebreak

\begin{homeworkProblem}
    The goal of this assignment is to introduce churn management using decision trees, logistic regression and neural network. You will try different combinations of the parameters to see their impacts on the accuracy of your models for this specific data set. This data set contains summarized data records for each customer for a phone company. Our goal is to build a model so that this company can predict potential churners. 
Two data sets are available, churn\_training.txt and churn\_validation.txt. Each data set has 21 variables. They are:

State:
Account\_length: how long this person has been in this plan

Area\_code:

Phone\_number:

International\_plan: this person has international plan=1, otherwise=0

Voice\_mail\_plan: this person has voice mail plan=1, otherwise=0

Number\_vmail\_messages: number of voice mails

Total\_day\_minutes:

Total\_day\_calls:

Total\_day\_charge:

Total\_eve\_minutes:

Total\_eve\_calls:

Total\_eve\_charge:

Total\_night\_minutes:

Total\_night\_calls:

Total\_night\_charge:

Total\_intl\_minutes:

Total\_intl\_calls:

Total\_intl\_charge:

Number\_customer\_service\_calls:

Class: churn=1, did not churn=0

Each row in “churn\_training” represents the customer record. The training data contains 2000 rows and the validation data contains 1033 records.  
\\\\1)Perform decision tree classification on training data set. Select all the input variables except state, area\_code, and phone\_number (since they are only informative for this analysis). Set the “Direction” of class as “out”, “type” as “Flag”. Then, specify the “minimum records per child branch” as 30, “pruning severity” as 60, click “use global pruning”. Hand-in the confusion matrices for validation data.
\\\\
\solution
 \begin{figure}[H]
  \centering
  \includegraphics[width=3in]{hwdt.eps}
  \caption{confusion matrix for decision tree.}
\end{figure}
2)Perform neural network on training data set using default settings. Again, select all the input variables except state, area\_code, and phone\_number. Hand-in the confusion matrix for validation data.
\\\\
\solution
 \begin{figure}[H]
  \centering
  \includegraphics[width=3in]{hwnn.eps}
  \caption{confusion matrix for neural network.}
\end{figure}
3)Perform logistic regression on training data set using default settings. Again, select all the input variables except state, area\_code, and phone\_number. Hand-in the confusion matrix for validation data.
\\\\
\solution
 \begin{figure}[H]
  \centering
  \includegraphics[width=3in]{hwld.eps}
  \caption{confusion matrix for logistic regression}
\end{figure}
4)Hand-in your observations on the model quality for decision tree, neural network and logistic regression using the confusion matrices.
\\\\
\solution
\\The confusion matrices for decision tree,neural network,logistic regression is shown in the Figure 5,6,7.Through my observations I found that decision tree get the best result.And the result of neural network is changed for different training and testing process.
\end{homeworkProblem}
\pagebreak

\end{document}
\texttt{\textit{}}